\documentclass[conference]{IEEEtran}
\usepackage{pifont}
\usepackage{times,amsmath,color,
amssymb,graphicx,epsfig,cite,psfrag,subfigure,algorithm,multirow,cases,balance}
\newtheorem{claim}{Claim}
\newtheorem{guess}{Conjecture}
\newtheorem{definition}{Definition}
\newtheorem{fact}{Fact}
\newtheorem{assumption}{Assumption}
\newtheorem{theorem}{\underline{Theorem}}
\newtheorem{lemma}{\underline{Lemma}}
\newtheorem{ctheorem}{Corrected Theorem}
\newtheorem{corollary}{\underline{Corollary}}
\newtheorem{proposition}{Proposition}
\newtheorem{example}{\underline{Example}}
\newtheorem{remark}{\underline{Remark}}
\newtheorem{problem}{\underline{Problem}}
\def\Ei{\mathop\mathrm{Ei}}
\def\E{\mathop\mathrm{E}}
\def\tr{\mathop\mathrm{tr}}
\newcounter{mytempeqncnt}
%\IEEEoverridecommandlockouts

\begin{document}

\title{Literature Survey \\ Unsupervised Learning and Deep Learning}
\author{This is me\\%
Language Technologies Institute, School of Computer Science, Pittsburgh, PA, 15213, USA\\
Email: myid@cs.cmu.edu
}

\thanks{This work was supported...}




\maketitle %\thispagestyle{empty}




\begin{abstract}
In this paper, we study ...

\end{abstract}



\section{Introduction}

TODO:

\section{Spectrum Sensing Strategy}





%\begin{figure}[t]
%\centering
%\includegraphics[width=70mm]{thresholds.eps}
%\caption{Multiple power level detection from multiple thresholds.} 
%\label{fig:thresholds}
%\end{figure}

\begin{remark}
Sample remark...
\end{remark}



\section{Simulations}\label{sec:simulation}

In this section



\section{Conclusions}

In this paper, we investigated a new 


\balance
\begin{thebibliography}{99}

\bibitem{S.Haykin} S. Haykin, ``Cognitive radio: brain-empowered wireless communications,'' \textit{IEEE J. Select. Areas Commun.}, vol. 23, pp.~201--220, Feb. 2005.

\bibitem{MatchedFilter} T. Yucek and H. Arslan, ``A survey of spectrum sensing algorithms for
cognitive radio applications,'' {\it IEEE Commun. Survey \&
Turtorials}, vol. 11, no. 1, First Quater, 2009.




\end{thebibliography}







%\begin{table}[t]\label{Tab:1}
%\centering \caption{$Required\ Interference\ Protection\
%Ratios$}
%\begin{tabular}{|c|c|c|}
%\hline
%Type of service & Channel & Interference \\
%& Offset & Protection D/U \\
%& & Ratio(dB) \\
%\hline
%Analog TV & Lower Adjacent & -14 \\
%\cline{2-3}
%& Co-channel & 34 \\
%\cline{2-3}
%& Upper Adjacent & -17 \\
%\hline
%Digital TV & Lower Adjacent & -28 \\
%\cline{2-3}
%& Co-channel & 23 \\
%\cline{2-3}
%& Upper Adjacent & -26 \\
%\hline
%\end{tabular}
%\end{table}








\end{document}
